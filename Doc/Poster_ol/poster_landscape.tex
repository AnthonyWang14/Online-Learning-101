\documentclass[a0paper,landscape,final]{baposter}

\usepackage{times}
\usepackage{calc}
\usepackage{graphicx}
\usepackage{amsmath}
\usepackage{amssymb}
\usepackage{relsize}
\usepackage{multirow}
\usepackage{bm}
\usepackage{algorithm}
\usepackage{algorithmic}
\usepackage{subfigure}
\usepackage{slashbox}

\usepackage{graphicx}
\usepackage{multicol}

\usepackage{pgfbaselayers}
\pgfdeclarelayer{background}
\pgfdeclarelayer{foreground}
\pgfsetlayers{background,main,foreground}

\usepackage{helvet}
%\usepackage{bookman}
\usepackage{palatino}

\usepackage{enumitem}


\newcommand{\captionfont}{\footnotesize}

%\selectcolormodel{cmyk}

\graphicspath{{fig/}}

%%%%%%%%%%%%%%%%%%%%%%%%%%%%%%%%%%%%%%%%%%%%%%%%%%%%%%%%%%%%%%%%%%%%%%%%%%%%%%%%
%%%% Some math symbols used in the text
%%%%%%%%%%%%%%%%%%%%%%%%%%%%%%%%%%%%%%%%%%%%%%%%%%%%%%%%%%%%%%%%%%%%%%%%%%%%%%%%
% Format
\newcommand{\Matrix}[1]{\begin{bmatrix} #1 \end{bmatrix}}
\newcommand{\Vector}[1]{\Matrix{#1}}
\newcommand*{\SET}[1]  {\ensuremath{\mathcal{#1}}}
\newcommand*{\MAT}[1]  {\ensuremath{\mathbf{#1}}}
\newcommand*{\VEC}[1]  {\ensuremath{\bm{#1}}}
\newcommand*{\CONST}[1]{\ensuremath{\mathit{#1}}}
\newcommand*{\norm}[1]{\mathopen\| #1 \mathclose\|}% use instead of $\|x\|$
\newcommand*{\abs}[1]{\mathopen| #1 \mathclose|}% use instead of $\|x\|$
\newcommand*{\absLR}[1]{\left| #1 \right|}% use instead of $\|x\|$

\def\norm#1{\mathopen\| #1 \mathclose\|}% use instead of $\|x\|$
\newcommand{\normLR}[1]{\left\| #1 \right\|}% use instead of $\|x\|$

%%%%%%%%%%%%%%%%%%%%%%%%%%%%%%%%%%%%%%%%%%%%%%%%%%%%%%%%%%%%%%%%%%%%%%%%%%%%%%%%
% Multicol Settings
%%%%%%%%%%%%%%%%%%%%%%%%%%%%%%%%%%%%%%%%%%%%%%%%%%%%%%%%%%%%%%%%%%%%%%%%%%%%%%%%
\setlength{\columnsep}{0.7em}
\setlength{\columnseprule}{0mm}


%%%%%%%%%%%%%%%%%%%%%%%%%%%%%%%%%%%%%%%%%%%%%%%%%%%%%%%%%%%%%%%%%%%%%%%%%%%%%%%%
% Save space in lists. Use this after the opening of the list
%%%%%%%%%%%%%%%%%%%%%%%%%%%%%%%%%%%%%%%%%%%%%%%%%%%%%%%%%%%%%%%%%%%%%%%%%%%%%%%%
\newcommand{\compresslist}{%
\setlength{\itemsep}{1pt}%
\setlength{\parskip}{0pt}%
\setlength{\parsep}{0pt}%
}


%%%%%%%%%%%%%%%%%%%%%%%%%%%%%%%%%%%%%%%%%%%%%%%%%%%%%%%%%%%%%%%%%%%%%%%%%%%%%%
%%% Begin of Document
%%%%%%%%%%%%%%%%%%%%%%%%%%%%%%%%%%%%%%%%%%%%%%%%%%%%%%%%%%%%%%%%%%%%%%%%%%%%%%

\begin{document}

%%%%%%%%%%%%%%%%%%%%%%%%%%%%%%%%%%%%%%%%%%%%%%%%%%%%%%%%%%%%%%%%%%%%%%%%%%%%%%
%%% Here starts the poster
%%%---------------------------------------------------------------------------
%%% Format it to your taste with the options
%%%%%%%%%%%%%%%%%%%%%%%%%%%%%%%%%%%%%%%%%%%%%%%%%%%%%%%%%%%%%%%%%%%%%%%%%%%%%%
\typeout{Poster Starts}
\background{
  \begin{tikzpicture}[remember picture,overlay]%
   \draw (current page.north west)+(-2em,-0em) node[anchor=north west] {\hspace{-2em}\includegraphics[height=1.1\textheight]{silhouettes_background}};
 \end{tikzpicture}%
}
\definecolor{blue}{RGB}{150,200,242}
\definecolor{darkblue}{RGB}{68,92,170}
\definecolor{brown}{RGB}{245,236,215}

\begin{poster}{
  % Show grid to help with alignment
  grid=false,
  % Column spacing
  colspacing=1em,
  % Color style
  %headerColorOne=cyan!20!white!90!black,
  %borderColor=cyan!30!white!90!black,
  %bgColorOne=cyan!10!white,
  headerColorOne=blue,
  borderColor=darkblue,
  bgColorOne=brown,
  boxColorOne=white,
  %bgColorOne=lighteryellow,
  %bgColorTwo=lightestyellow,
  %borderColor=reddishyellow,
  %headerColorOne=yellow,
  %headerColorTwo=reddishyellow,
  %headerFontColor=black,
  %boxColorOne=lightyellow,
  %boxColorTwo=lighteryellow,
  % Format of textbox
  textborder=roundedleft,
  % Format of text header
  eyecatcher=true,
  headerborder=open,
  headerheight=0.08\textheight,
  headershape=roundedright,
  headershade=plain,
  headerfont=\Large\textsf, %Sans Serif
  boxshade=plain,
%  background=shade-tb,
  background=plain,
  linewidth=2pt,
  columns=3
  }
  % Eye Catcher
  {{\begin{minipage}{1.5cm}
    \vspace{0.5cm}
	\includegraphics[width=5cm]{ethzlogo}
\end{minipage}}
  } % No eye catcher for this poster. If an eye catcher is present, the title is centered between eye-catcher and logo.
  % Title
  {\sf %Sans Serif
  %\bf% Serif
  {\huge Online Learning with Expert Advice for Sports Betting}}
  % Authors
  {\sf %Sans Serif
  % Serif
  Matteo Turchetta\hspace{1cm}
  Riccardo Moriconi\hspace{1cm}
  Yijun Pan\\
  }
%  {{\begin{minipage}{1.5cm}	\includegraphics[width=2.5cm]{fig/usclogo.pdf}
%    \end{minipage}}
%  }
  % University logo
  
  \tikzstyle{light shaded}=[top color=baposterBGtwo!30!white,bottom color=baposterBGone!30!white,shading=axis,shading angle=30]

  % Width of left inset image
     \newlength{\leftimgwidth}
     \setlength{\leftimgwidth}{0.78em+8.0em}

%%%%%%%%%%%%%%%%%%%%%%%%%%%%%%%%%%%%%%%%%%%%%%%%%%%%%%%%%%%%%%%%%%%%%%%%%%%%%%
%%% Now define the boxes that make up the poster
%%%---------------------------------------------------------------------------
%%% Each box has a name and can be placed absolutely or relatively.
%%% The only inconvenience is that you can only specify a relative position
%%% towards an already declared box. So if you have a box attached to the
%%% bottom, one to the top and a third one which should be in between, you
%%% have to specify the top and bottom boxes before you specify the middle
%%% box.
%%%%%%%%%%%%%%%%%%%%%%%%%%%%%%%%%%%%%%%%%%%%%%%%%%%%%%%%%%%%%%%%%%%%%%%%%%%%%%
    %
    % A coloured circle useful as a bullet with an adjustably strong filling
    \newcommand{\colouredcircle}[1]{%
      \tikz{\useasboundingbox (-0.2em,-0.32em) rectangle(0.2em,0.32em); \draw[draw=black,fill=baposterBGone!80!black!#1!white,line width=0.03em] (0,0) circle(0.18em);}}

%%%%%%%%%%%%%%%%%%%%%%%%%%%%%%%%%%%%%%%%%%%%%%%%%%%%%%%%%%%%%%%%%%%%%%%%%%%%%%
  \headerbox{Overview}{name=overview,column=0,row=0}{
%%%%%%%%%%%%%%%%%%%%%%%%%%%%%%%%%%%%%%%%%%%%%%%%%%%%%%%%%%%%%%%%%%%%%%%%%%%%%%
{\bf Goal}

Apply online learning algorithms to derive profitable sport bets without prior knowledge of the games.

\vspace{0.07cm}
{\bf Online Learning with Expert Advice}
\vspace{-0.2cm}
\begin{itemize}[leftmargin=*]\compresslist
    %\setlength{\itemindent}{-1em}
	\item[-] The learner is given a pool of experts ({\bf bookmakers}) which provide advices in the form of forecasts ({\bf odds}) over the outcome of an event ({\bf sport game}). 
	\item[-] Experts are assigned weights based on their credits. 
	\item[-] The concept of credibility is defined on the loss function.
\end{itemize}
  
  
  }

%%%%%%%%%%%%%%%%%%%%%%%%%%%%%%%%%%%%%%%%%%%%%%%%%%%%%%%%%%%%%%%%%%%%%%%%%%%%%%
  \headerbox{Learning Algorithm}{name=algorithm, column=0,below=overview}{
%%%%%%%%%%%%%%%%%%%%%%%%%%%%%%%%%%%%%%%%%%%%%%%%%%%%%%%%%%%%%%%%%%%%%%%%%%%%%%
 \vspace{-0.4cm}
\begin{algorithm}[H]
\caption{Exponential Weight Algorithms}
\begin{algorithmic}[1]
 \STATE ${w_0^k} = 1/K, k = 1,...,K$
\FORALL {$t = 1:N$}
    \STATE Experts announce predictions $\gamma _t^k \in \Gamma ,k = 1,...,K$
    \STATE Algorithm announces ${\gamma _t} = S(\gamma _t^1,...,\gamma _t^K) \in \Gamma$
    \STATE Reality announces ${\omega _t} \in \Omega$
    \STATE $w_t^k = w_{t - 1}^k{e^{ - \eta \lambda ({\omega _N},\lambda _t^k)}}$
\ENDFOR
\STATE Return ${\bm{x}}^{(M)}, {\bm{y}}^{(M)}$
\end{algorithmic}
\label{alg:inf}
\end{algorithm}
\vspace{-0.7cm}
\begin{itemize}[leftmargin = *]\compresslist
    \item[-] For Weighted Average Algorithm, $S$ is weighted mean.
    \item[-] For Follow the Leader Algorithm, $S$ is maximization.
    \item[-] For Strong Aggregating Algorithm, a substitution function $S'(g_t)$ is applied on the generalised prediction ${g_t}(\omega )$.
\end{itemize}
\vspace{-0.3cm}
 

  }
  %%%%%%%%%%%%%%%%%%%%%%%%%%%%%%%%%%%%%%%%%%%%%%%%%%%%%%%%%%%%%%%%%%%%%%%%%%%%%%
  \headerbox{Loss Function}{name=loss, column=0, below=algorithm, above=bottom}{
%%%%%%%%%%%%%%%%%%%%%%%%%%%%%%%%%%%%%%%%%%%%%%%%%%%%%%%%%%%%%%%%%%%%%%%%%%%%%%
The proposed loss functions are minimised when the predicted distribution coincides with the true probability distribution ( {\bf Proper Scoring Rule}).

\vspace{0.1cm}
{\bf Brier Loss}

In this case the target distribution ${\delta _\omega}$ is a Dirac distribution concentrated at the actual outcome y of the match.
\[\ell \left( {\gamma ,\underline \omega  } \right) = \sum\limits_{\omega  \in \Omega } {{{\left( {\gamma \left( \omega  \right) - {\delta _{\underline \omega  }}\left( \omega  \right)} \right)}^2}} \]
{\bf Calibration Loss}

A forecaster is $\epsilon$-calibrated if:
\[P\left( {\omega  = \omega_*|\gamma \left( \omega_* \right) \in \left[ {p - \varepsilon ;p + \varepsilon } \right]} \right) \approx p,\varepsilon  > 0\]

In practise, reality does not reveal the true probability distribution. But we can approximate through relative frequency $ $ and use it as target distribution. 

  }
  
    %%%%%%%%%%%%%%%%%%%%%%%%%%%%%%%%%%%%%%%%%%%%%%%%%%%%%%%%%%%%%%%%%%%%%%%%%%%%%%
  \headerbox{Loss Function}{name=losscont, column=1}{
%%%%%%%%%%%%%%%%%%%%%%%%%%%%%%%%%%%%%%%%%%%%%%%%%%%%%%%%%%%%%%%%%%%%%%%%%%%%%%
Thus the calibration loss function can be defined as
\[P\left( {\omega  = \omega_*|\gamma \left( i \right) \in \left[ {p - \varepsilon ;p + \varepsilon } \right]} \right) \approx p,\varepsilon  > 0\]
Multiple loss functions can be derived by replacing the $l2$ norm.
\vspace{-0.2cm}
\begin{itemize}[leftmargin = *]\compresslist
    \item[-] KL divergence (logarithmic loss)
    \item[-] $l1$ norm
\end{itemize}
\vspace{-0.3cm}

  }
  
%%%%%%%%%%%%%%%%%%%%%%%%%%%%%%%%%%%%%%%%%%%%%%%%%%%%%%%%%%%%%%%%%%%%%%%%%%%%%%
  \headerbox{Results}{name=results,column=2,span=1,row=0}{
%%%%%%%%%%%%%%%%%%%%%%%%%%%%%%%%%%%%%%%%%%%%%%%%%%%%%%%%%%%%%%%%%%%%%%%%%%%%%%
  }

%%%%%%%%%%%%%%%%%%%%%%%%%%%%%%%%%%%%%%%%%%%%%%%%%%%%%%%%%%%%%%%%%%%%%%%%%%%%%%
  \headerbox{Conclusion}{name=conclusion,column=2,below=results, span=1,row=0}{
%%%%%%%%%%%%%%%%%%%%%%%%%%%%%%%%%%%%%%%%%%%%%%%%%%%%%%%%%%%%%%%%%%%%%%%%%%%%%%
  }


\end{poster}%
%
\end{document}
